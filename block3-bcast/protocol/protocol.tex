\documentclass[a4paper,10pt]{article}

\usepackage{amsmath}
\usepackage{amssymb}
\usepackage[usenames,dvipsnames]{color}
\usepackage{comment}
\usepackage[utf8]{inputenc}
\usepackage{listings}
\usepackage[pdfborder={0 0 0}]{hyperref}
\usepackage{spverbatim}
\usepackage{booktabs}
\usepackage{graphicx}

\definecolor{OliveGreen}{cmyk}{0.64,0,0.95,0.40}
\definecolor{Gray}{gray}{0.5}

\lstset{
    language=C,
    basicstyle=\ttfamily,
    keywordstyle=\color{OliveGreen},
    commentstyle=\color{Gray},
    captionpos=b,
    breaklines=true,
    breakatwhitespace=false,
    showspaces=false,
    showtabs=false,
    numbers=left,
}

\title{VU High Performance Computing \\
       SS 2013 \\
       Block 3: MPI Library Implementation \\
       Broadcast Algorithms}
\author{Jakob Gruber, 0203440}

\begin{document}

\maketitle
\tableofcontents
\pagebreak

\section{Introduction} \label{section:introduction}

In this assignment, we implemented three different broadcast algorithms and
subsequently benchmarked and compared these against the \lstinline|MPI_Bcast|
implementation in NEC MPI.

\section{Results}

The following results were obtained by benchmarking with NECmpi on the Jupiter system. The
application was compiled using gcc 4.4.7 with

\begin{verbatim}
CFLAGS = -Wall -Wextra -pedantic -std=c99 -D_GNU_SOURCE -O3
\end{verbatim}

A benchmark run can be started by using the \verb|make bench| and \verb|make bench-bs|
targets in our makefile. Results are written to \verb|results/bench.csv|.

\begin{comment}
TODO: "Estimate best block size using measured alpha and beta"??
TODO: "Verify/falsify claims of lecture"
TODO: "Estimate parameter ranges (processes, data sizes) for which algorithms are ideal.
TODO: Random communicator??
TODO: How to avoid performance problems (random communicators), which means MPI provides (NONE?)
Show graphs.
Briefly describe algorithms.
Best block sizes for linear/binary.
\end{comment}


\section{Analysis}

\end{document}
